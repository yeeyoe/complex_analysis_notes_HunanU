% Options for packages loaded elsewhere
% Options for packages loaded elsewhere
\PassOptionsToPackage{unicode}{hyperref}
\PassOptionsToPackage{hyphens}{url}
\PassOptionsToPackage{dvipsnames,svgnames,x11names}{xcolor}
%
\documentclass[
  chinese-hans,
  letterpaper,
]{ctexbook}
\usepackage{xcolor}
\usepackage{amsmath,amssymb}
\setcounter{secnumdepth}{5}
\usepackage{iftex}
\ifPDFTeX
  \usepackage[T1]{fontenc}
  \usepackage[utf8]{inputenc}
  \usepackage{textcomp} % provide euro and other symbols
\else % if luatex or xetex
  \usepackage{unicode-math} % this also loads fontspec
  \defaultfontfeatures{Scale=MatchLowercase}
  \defaultfontfeatures[\rmfamily]{Ligatures=TeX,Scale=1}
\fi
\usepackage{lmodern}
\ifPDFTeX\else
  % xetex/luatex font selection
\fi
% Use upquote if available, for straight quotes in verbatim environments
\IfFileExists{upquote.sty}{\usepackage{upquote}}{}
\IfFileExists{microtype.sty}{% use microtype if available
  \usepackage[]{microtype}
  \UseMicrotypeSet[protrusion]{basicmath} % disable protrusion for tt fonts
}{}
\makeatletter
\@ifundefined{KOMAClassName}{% if non-KOMA class
  \IfFileExists{parskip.sty}{%
    \usepackage{parskip}
  }{% else
    \setlength{\parindent}{0pt}
    \setlength{\parskip}{6pt plus 2pt minus 1pt}}
}{% if KOMA class
  \KOMAoptions{parskip=half}}
\makeatother
% Make \paragraph and \subparagraph free-standing
\makeatletter
\ifx\paragraph\undefined\else
  \let\oldparagraph\paragraph
  \renewcommand{\paragraph}{
    \@ifstar
      \xxxParagraphStar
      \xxxParagraphNoStar
  }
  \newcommand{\xxxParagraphStar}[1]{\oldparagraph*{#1}\mbox{}}
  \newcommand{\xxxParagraphNoStar}[1]{\oldparagraph{#1}\mbox{}}
\fi
\ifx\subparagraph\undefined\else
  \let\oldsubparagraph\subparagraph
  \renewcommand{\subparagraph}{
    \@ifstar
      \xxxSubParagraphStar
      \xxxSubParagraphNoStar
  }
  \newcommand{\xxxSubParagraphStar}[1]{\oldsubparagraph*{#1}\mbox{}}
  \newcommand{\xxxSubParagraphNoStar}[1]{\oldsubparagraph{#1}\mbox{}}
\fi
\makeatother

\usepackage{color}
\usepackage{fancyvrb}
\newcommand{\VerbBar}{|}
\newcommand{\VERB}{\Verb[commandchars=\\\{\}]}
\DefineVerbatimEnvironment{Highlighting}{Verbatim}{commandchars=\\\{\}}
% Add ',fontsize=\small' for more characters per line
\usepackage{framed}
\definecolor{shadecolor}{RGB}{241,243,245}
\newenvironment{Shaded}{\begin{snugshade}}{\end{snugshade}}
\newcommand{\AlertTok}[1]{\textcolor[rgb]{0.68,0.00,0.00}{#1}}
\newcommand{\AnnotationTok}[1]{\textcolor[rgb]{0.37,0.37,0.37}{#1}}
\newcommand{\AttributeTok}[1]{\textcolor[rgb]{0.40,0.45,0.13}{#1}}
\newcommand{\BaseNTok}[1]{\textcolor[rgb]{0.68,0.00,0.00}{#1}}
\newcommand{\BuiltInTok}[1]{\textcolor[rgb]{0.00,0.23,0.31}{#1}}
\newcommand{\CharTok}[1]{\textcolor[rgb]{0.13,0.47,0.30}{#1}}
\newcommand{\CommentTok}[1]{\textcolor[rgb]{0.37,0.37,0.37}{#1}}
\newcommand{\CommentVarTok}[1]{\textcolor[rgb]{0.37,0.37,0.37}{\textit{#1}}}
\newcommand{\ConstantTok}[1]{\textcolor[rgb]{0.56,0.35,0.01}{#1}}
\newcommand{\ControlFlowTok}[1]{\textcolor[rgb]{0.00,0.23,0.31}{\textbf{#1}}}
\newcommand{\DataTypeTok}[1]{\textcolor[rgb]{0.68,0.00,0.00}{#1}}
\newcommand{\DecValTok}[1]{\textcolor[rgb]{0.68,0.00,0.00}{#1}}
\newcommand{\DocumentationTok}[1]{\textcolor[rgb]{0.37,0.37,0.37}{\textit{#1}}}
\newcommand{\ErrorTok}[1]{\textcolor[rgb]{0.68,0.00,0.00}{#1}}
\newcommand{\ExtensionTok}[1]{\textcolor[rgb]{0.00,0.23,0.31}{#1}}
\newcommand{\FloatTok}[1]{\textcolor[rgb]{0.68,0.00,0.00}{#1}}
\newcommand{\FunctionTok}[1]{\textcolor[rgb]{0.28,0.35,0.67}{#1}}
\newcommand{\ImportTok}[1]{\textcolor[rgb]{0.00,0.46,0.62}{#1}}
\newcommand{\InformationTok}[1]{\textcolor[rgb]{0.37,0.37,0.37}{#1}}
\newcommand{\KeywordTok}[1]{\textcolor[rgb]{0.00,0.23,0.31}{\textbf{#1}}}
\newcommand{\NormalTok}[1]{\textcolor[rgb]{0.00,0.23,0.31}{#1}}
\newcommand{\OperatorTok}[1]{\textcolor[rgb]{0.37,0.37,0.37}{#1}}
\newcommand{\OtherTok}[1]{\textcolor[rgb]{0.00,0.23,0.31}{#1}}
\newcommand{\PreprocessorTok}[1]{\textcolor[rgb]{0.68,0.00,0.00}{#1}}
\newcommand{\RegionMarkerTok}[1]{\textcolor[rgb]{0.00,0.23,0.31}{#1}}
\newcommand{\SpecialCharTok}[1]{\textcolor[rgb]{0.37,0.37,0.37}{#1}}
\newcommand{\SpecialStringTok}[1]{\textcolor[rgb]{0.13,0.47,0.30}{#1}}
\newcommand{\StringTok}[1]{\textcolor[rgb]{0.13,0.47,0.30}{#1}}
\newcommand{\VariableTok}[1]{\textcolor[rgb]{0.07,0.07,0.07}{#1}}
\newcommand{\VerbatimStringTok}[1]{\textcolor[rgb]{0.13,0.47,0.30}{#1}}
\newcommand{\WarningTok}[1]{\textcolor[rgb]{0.37,0.37,0.37}{\textit{#1}}}

\usepackage{longtable,booktabs,array}
\usepackage{calc} % for calculating minipage widths
% Correct order of tables after \paragraph or \subparagraph
\usepackage{etoolbox}
\makeatletter
\patchcmd\longtable{\par}{\if@noskipsec\mbox{}\fi\par}{}{}
\makeatother
% Allow footnotes in longtable head/foot
\IfFileExists{footnotehyper.sty}{\usepackage{footnotehyper}}{\usepackage{footnote}}
\makesavenoteenv{longtable}
\usepackage{graphicx}
\makeatletter
\newsavebox\pandoc@box
\newcommand*\pandocbounded[1]{% scales image to fit in text height/width
  \sbox\pandoc@box{#1}%
  \Gscale@div\@tempa{\textheight}{\dimexpr\ht\pandoc@box+\dp\pandoc@box\relax}%
  \Gscale@div\@tempb{\linewidth}{\wd\pandoc@box}%
  \ifdim\@tempb\p@<\@tempa\p@\let\@tempa\@tempb\fi% select the smaller of both
  \ifdim\@tempa\p@<\p@\scalebox{\@tempa}{\usebox\pandoc@box}%
  \else\usebox{\pandoc@box}%
  \fi%
}
% Set default figure placement to htbp
\def\fps@figure{htbp}
\makeatother



\ifLuaTeX
\usepackage[bidi=basic,provide=*]{babel}
\else
\usepackage[bidi=default,provide=*]{babel}
\fi
% get rid of language-specific shorthands (see #6817):
\let\LanguageShortHands\languageshorthands
\def\languageshorthands#1{}


\setlength{\emergencystretch}{3em} % prevent overfull lines

\providecommand{\tightlist}{%
  \setlength{\itemsep}{0pt}\setlength{\parskip}{0pt}}



 


\usepackage{xeCJK}
\newcommand{\bb}[1]{\left( #1 \right)}
\newcommand{\res}[2]{\mathrm{Res}_{#1}\left( #2 \right)}
\makeatletter
\@ifpackageloaded{bookmark}{}{\usepackage{bookmark}}
\makeatother
\makeatletter
\@ifpackageloaded{caption}{}{\usepackage{caption}}
\AtBeginDocument{%
\ifdefined\contentsname
  \renewcommand*\contentsname{目录}
\else
  \newcommand\contentsname{目录}
\fi
\ifdefined\listfigurename
  \renewcommand*\listfigurename{图索引}
\else
  \newcommand\listfigurename{图索引}
\fi
\ifdefined\listtablename
  \renewcommand*\listtablename{表索引}
\else
  \newcommand\listtablename{表索引}
\fi
\ifdefined\figurename
  \renewcommand*\figurename{图}
\else
  \newcommand\figurename{图}
\fi
\ifdefined\tablename
  \renewcommand*\tablename{表}
\else
  \newcommand\tablename{表}
\fi
}
\@ifpackageloaded{float}{}{\usepackage{float}}
\floatstyle{ruled}
\@ifundefined{c@chapter}{\newfloat{codelisting}{h}{lop}}{\newfloat{codelisting}{h}{lop}[chapter]}
\floatname{codelisting}{列表}
\newcommand*\listoflistings{\listof{codelisting}{列表索引}}
\usepackage{amsthm}
\theoremstyle{definition}
\newtheorem{example}{例}[chapter]
\theoremstyle{plain}
\newtheorem{corollary}{推论}[chapter]
\theoremstyle{plain}
\newtheorem{theorem}{定理}[chapter]
\theoremstyle{definition}
\newtheorem{definition}{定义}[chapter]
\theoremstyle{definition}
\newtheorem{exercise}{练习}[chapter]
\theoremstyle{remark}
\AtBeginDocument{\renewcommand*{\proofname}{证明}}
\newtheorem*{remark}{注记}
\newtheorem*{solution}{解}
\newtheorem{refremark}{注记}[chapter]
\newtheorem{refsolution}{解}[chapter]
\makeatother
\makeatletter
\makeatother
\makeatletter
\@ifpackageloaded{caption}{}{\usepackage{caption}}
\@ifpackageloaded{subcaption}{}{\usepackage{subcaption}}
\makeatother
\usepackage{bookmark}
\IfFileExists{xurl.sty}{\usepackage{xurl}}{} % add URL line breaks if available
\urlstyle{same}
\hypersetup{
  pdftitle={湖南大学复变函数讲义},
  pdfauthor={姚懿},
  pdflang={zh-Hans},
  colorlinks=true,
  linkcolor={blue},
  filecolor={Maroon},
  citecolor={Blue},
  urlcolor={Blue},
  pdfcreator={LaTeX via pandoc}}


\title{湖南大学复变函数讲义}
\author{姚懿}
\date{2026-02-26}
\begin{document}
\frontmatter
\maketitle

\renewcommand*\contentsname{目录}
{
\hypersetup{linkcolor=}
\setcounter{tocdepth}{2}
\tableofcontents
}

\mainmatter
\bookmarksetup{startatroot}

\chapter*{前言}\label{ux524dux8a00}
\addcontentsline{toc}{chapter}{前言}

\markboth{前言}{前言}

这是复变函数的讲义。

\bookmarksetup{startatroot}

\chapter{复数}\label{ux590dux6570}

这一章我们主要介绍复数的概念、复数域、复数的几何意义、复数的运算法则等。

\section{复数的定义}\label{ux590dux6570ux7684ux5b9aux4e49}

复数的引入是为了推广实数域到复域,以便于描述和研究一些在实数域上不存在的量,例如根为负数的二次方程的解。

\begin{definition}[]\protect\hypertarget{def-complex-field}{}\label{def-complex-field}

\textbf{(复数域)}\\
复数域 \(\mathbb{C}\) 定义为有序实数对 \((a,b)\) 的集合,配备加法\\
\begin{equation}\phantomsection\label{eq-complex-add}{(a,b)+(c,d)=(a+c,b+d)}\end{equation}
与乘法\\
\begin{equation}\phantomsection\label{eq-complex-mul}{(a,b)\cdot(c,d)=(ac-bd,ad+bc).}\end{equation}

\end{definition}

上述 定义~\ref{def-complex-field} 是标准的。

\begin{theorem}[]\protect\hypertarget{thm-complex-field}{}\label{thm-complex-field}

在运算(\ref{eq-complex-add})和(\ref{eq-complex-mul})下,\(\mathbb{C}\)
构成一个域,其零元为 \((0,0)\),单位元为
\((1,0)\),并满足交换律、结合律及分配律。

\end{theorem}

\begin{proof}
回忆 域 的定义,验证上述运算满足所有域的性质。留作习题。
\end{proof}

\begin{corollary}[]\protect\hypertarget{cor-complex-field}{}\label{cor-complex-field}

复数域 \(\mathbb{C}\) 中非零元素均可逆。

\end{corollary}

\begin{example}[]\protect\hypertarget{exm-complex-field}{}\label{exm-complex-field}

由(\ref{eq-complex-mul}),计算 \((2+3i)(-1-4i)=(2-3i)(-1+4i)=-10+13i\)。

\end{example}

\begin{example}[]\protect\hypertarget{exm-complex-field-2}{}\label{exm-complex-field-2}

由 推论~\ref{cor-complex-field},计算复数的逆元 \((2+3i)^{-1}=2-3i\)。

\end{example}

\begin{exercise}[]\protect\hypertarget{exr-complex-field-2}{}\label{exr-complex-field-2}

验证 \((2+3i)^{-1}=2-3i\)。

\end{exercise}

\[\res{a}{f(z)}\] For a demonstration of a line plot on a polar axis,
see 图~\ref{fig-polar}.

\begin{Shaded}
\begin{Highlighting}[]
\ImportTok{import}\NormalTok{ numpy }\ImportTok{as}\NormalTok{ np}
\ImportTok{import}\NormalTok{ matplotlib.pyplot }\ImportTok{as}\NormalTok{ plt}

\NormalTok{r }\OperatorTok{=}\NormalTok{ np.arange(}\DecValTok{0}\NormalTok{, }\DecValTok{2}\NormalTok{, }\FloatTok{0.01}\NormalTok{)}
\NormalTok{theta }\OperatorTok{=} \DecValTok{2} \OperatorTok{*}\NormalTok{ np.pi }\OperatorTok{*}\NormalTok{ r}
\NormalTok{fig, ax }\OperatorTok{=}\NormalTok{ plt.subplots(}
\NormalTok{  subplot\_kw }\OperatorTok{=}\NormalTok{ \{}\StringTok{\textquotesingle{}projection\textquotesingle{}}\NormalTok{: }\StringTok{\textquotesingle{}polar\textquotesingle{}}\NormalTok{\} }
\NormalTok{)}
\NormalTok{ax.plot(theta, r)}
\NormalTok{ax.set\_rticks([}\FloatTok{0.5}\NormalTok{, }\DecValTok{1}\NormalTok{, }\FloatTok{1.5}\NormalTok{, }\DecValTok{2}\NormalTok{])}
\NormalTok{ax.grid(}\VariableTok{True}\NormalTok{)}
\NormalTok{plt.show()}
\end{Highlighting}
\end{Shaded}

\begin{figure}[H]

\centering{

\pandocbounded{\includegraphics[keepaspectratio]{chapter1/Chap1_index_files/figure-pdf/fig-polar-output-1.pdf}}

}

\caption{\label{fig-polar}A line plot on a polar axis}

\end{figure}%

\section{复数的几何意义}\label{ux590dux6570ux7684ux51e0ux4f55ux610fux4e49}

这里是几何意义的内容。

回顾 定义~\ref{def-complex-field} 的定义。 @ref(def-complex-field)

\subsection{复数乘法示意图}\label{ux590dux6570ux4e58ux6cd5ux793aux610fux56fe}

下图展示了复数乘法的几何意义:模相乘,辐角相加。

\begin{figure}

\centering{

\pandocbounded{\includegraphics[keepaspectratio]{chapter1/Chap1_index_files/figure-pdf/fig-complex-mult-output-1.pdf}}

}

\caption{\label{fig-complex-mult}复数乘法示意图}

\end{figure}%

\section{习题}\label{ux4e60ux9898}

这里是第一章的习题。

\bookmarksetup{startatroot}

\chapter{解析函数}\label{ux89e3ux6790ux51fdux6570}

\section{解析函数}\label{ux89e3ux6790ux51fdux6570-1}

这里是解析函数的内容。回忆(定理~\ref{thm-complex-field})和
推论~\ref{cor-complex-field}。

\begin{definition}[]\protect\hypertarget{def-holo-function}{}\label{def-holo-function}

一个函数 \(f\) 是holomorphic的,如果它在其域上是可微的。

\end{definition}

注意这里的差商是关于复数的除法(\ref{eq-complex-mul})。

\section{柯西-黎曼方程}\label{ux67efux897f-ux9eceux66fcux65b9ux7a0b}

这里是柯西-黎曼方程的内容。

\section{习题}\label{ux4e60ux9898-1}

这里是第二章的习题。


\backmatter


\end{document}
